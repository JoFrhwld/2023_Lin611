% Options for packages loaded elsewhere
\PassOptionsToPackage{unicode}{hyperref}
\PassOptionsToPackage{hyphens}{url}
\PassOptionsToPackage{dvipsnames,svgnames,x11names}{xcolor}
%
\documentclass[
  letterpaper,
  DIV=11,
  numbers=noendperiod]{scrartcl}

\usepackage{amsmath,amssymb}
\usepackage{lmodern}
\usepackage{iftex}
\ifPDFTeX
  \usepackage[T1]{fontenc}
  \usepackage[utf8]{inputenc}
  \usepackage{textcomp} % provide euro and other symbols
\else % if luatex or xetex
  \usepackage{unicode-math}
  \defaultfontfeatures{Scale=MatchLowercase}
  \defaultfontfeatures[\rmfamily]{Ligatures=TeX,Scale=1}
  \setmainfont[]{Atkinson Hyperlegible}
\fi
% Use upquote if available, for straight quotes in verbatim environments
\IfFileExists{upquote.sty}{\usepackage{upquote}}{}
\IfFileExists{microtype.sty}{% use microtype if available
  \usepackage[]{microtype}
  \UseMicrotypeSet[protrusion]{basicmath} % disable protrusion for tt fonts
}{}
\makeatletter
\@ifundefined{KOMAClassName}{% if non-KOMA class
  \IfFileExists{parskip.sty}{%
    \usepackage{parskip}
  }{% else
    \setlength{\parindent}{0pt}
    \setlength{\parskip}{6pt plus 2pt minus 1pt}}
}{% if KOMA class
  \KOMAoptions{parskip=half}}
\makeatother
\usepackage{xcolor}
\usepackage[normalem]{ulem}
\setlength{\emergencystretch}{3em} % prevent overfull lines
\setcounter{secnumdepth}{1}
% Make \paragraph and \subparagraph free-standing
\ifx\paragraph\undefined\else
  \let\oldparagraph\paragraph
  \renewcommand{\paragraph}[1]{\oldparagraph{#1}\mbox{}}
\fi
\ifx\subparagraph\undefined\else
  \let\oldsubparagraph\subparagraph
  \renewcommand{\subparagraph}[1]{\oldsubparagraph{#1}\mbox{}}
\fi


\providecommand{\tightlist}{%
  \setlength{\itemsep}{0pt}\setlength{\parskip}{0pt}}\usepackage{longtable,booktabs,array}
\usepackage{calc} % for calculating minipage widths
% Correct order of tables after \paragraph or \subparagraph
\usepackage{etoolbox}
\makeatletter
\patchcmd\longtable{\par}{\if@noskipsec\mbox{}\fi\par}{}{}
\makeatother
% Allow footnotes in longtable head/foot
\IfFileExists{footnotehyper.sty}{\usepackage{footnotehyper}}{\usepackage{footnote}}
\makesavenoteenv{longtable}
\usepackage{graphicx}
\makeatletter
\def\maxwidth{\ifdim\Gin@nat@width>\linewidth\linewidth\else\Gin@nat@width\fi}
\def\maxheight{\ifdim\Gin@nat@height>\textheight\textheight\else\Gin@nat@height\fi}
\makeatother
% Scale images if necessary, so that they will not overflow the page
% margins by default, and it is still possible to overwrite the defaults
% using explicit options in \includegraphics[width, height, ...]{}
\setkeys{Gin}{width=\maxwidth,height=\maxheight,keepaspectratio}
% Set default figure placement to htbp
\makeatletter
\def\fps@figure{htbp}
\makeatother

\KOMAoption{captions}{tableheading}
\makeatletter
\makeatother
\makeatletter
\makeatother
\makeatletter
\@ifpackageloaded{caption}{}{\usepackage{caption}}
\AtBeginDocument{%
\ifdefined\contentsname
  \renewcommand*\contentsname{Table of contents}
\else
  \newcommand\contentsname{Table of contents}
\fi
\ifdefined\listfigurename
  \renewcommand*\listfigurename{List of Figures}
\else
  \newcommand\listfigurename{List of Figures}
\fi
\ifdefined\listtablename
  \renewcommand*\listtablename{List of Tables}
\else
  \newcommand\listtablename{List of Tables}
\fi
\ifdefined\figurename
  \renewcommand*\figurename{Figure}
\else
  \newcommand\figurename{Figure}
\fi
\ifdefined\tablename
  \renewcommand*\tablename{Table}
\else
  \newcommand\tablename{Table}
\fi
}
\@ifpackageloaded{float}{}{\usepackage{float}}
\floatstyle{ruled}
\@ifundefined{c@chapter}{\newfloat{codelisting}{h}{lop}}{\newfloat{codelisting}{h}{lop}[chapter]}
\floatname{codelisting}{Listing}
\newcommand*\listoflistings{\listof{codelisting}{List of Listings}}
\makeatother
\makeatletter
\@ifpackageloaded{caption}{}{\usepackage{caption}}
\@ifpackageloaded{subcaption}{}{\usepackage{subcaption}}
\makeatother
\makeatletter
\@ifpackageloaded{tcolorbox}{}{\usepackage[many]{tcolorbox}}
\makeatother
\makeatletter
\@ifundefined{shadecolor}{\definecolor{shadecolor}{rgb}{.97, .97, .97}}
\makeatother
\makeatletter
\makeatother
\ifLuaTeX
  \usepackage{selnolig}  % disable illegal ligatures
\fi
\IfFileExists{bookmark.sty}{\usepackage{bookmark}}{\usepackage{hyperref}}
\IfFileExists{xurl.sty}{\usepackage{xurl}}{} % add URL line breaks if available
\urlstyle{same} % disable monospaced font for URLs
\hypersetup{
  pdftitle={Lin611-001: Quantitative Methods in Linguistics},
  pdfauthor={Josef Fruehwald},
  colorlinks=true,
  linkcolor={blue},
  filecolor={Maroon},
  citecolor={Blue},
  urlcolor={Blue},
  pdfcreator={LaTeX via pandoc}}

\title{Lin611-001: Quantitative Methods in Linguistics}
\author{Josef Fruehwald}
\date{January 2023}

\begin{document}
\maketitle
\ifdefined\Shaded\renewenvironment{Shaded}{\begin{tcolorbox}[borderline west={3pt}{0pt}{shadecolor}, enhanced, breakable, interior hidden, boxrule=0pt, frame hidden, sharp corners]}{\end{tcolorbox}}\fi

\renewcommand*\contentsname{Sections}
{
\hypersetup{linkcolor=}
\setcounter{tocdepth}{1}
\tableofcontents
}
\hypertarget{key-info}{%
\section{Key info}\label{key-info}}

\begin{figure}

\begin{minipage}[t]{0.50\linewidth}

{\centering 

\hypertarget{where-and-when}{%
\subsection{Where and When}\label{where-and-when}}

\begin{longtable}[]{@{}ll@{}}
\toprule()
\endhead
Where: & Funkhouser, 307B \\
When: & Tuesdays \& Thursdays, 12:30 : 13:45 \\
\bottomrule()
\end{longtable}

}

\end{minipage}%
%
\begin{minipage}[t]{0.50\linewidth}

{\centering 

\hypertarget{instructor}{%
\subsection{Instructor}\label{instructor}}

\begin{longtable}[]{@{}ll@{}}
\toprule()
\endhead
& Dr.~Josef Fruehwald \\
email: & josef.fruehwald@uky.edu \\
office hours: & Mondays, 14:00 - 15:00 \\
office location: & POT 1671 \\
\bottomrule()
\end{longtable}

}

\end{minipage}%

\end{figure}

\begin{center}\rule{0.5\linewidth}{0.5pt}\end{center}

\hypertarget{course-at-a-glance}{%
\section{Course at a Glance}\label{course-at-a-glance}}

\begin{description}
\item[What you'll learn:]
the basics of statistical reasoning, linear modelling, data organization
\& visualization, R
\item[What you'll do:]
in-class exercises, a textbook work-through ``blog'', a midterm project,
a final project.
\item[What you'll need:]
the course textbook, a computing device with a physical keyboard
\item[The final-est deadline]
\protect\hyperlink{the-final-est-deadline}{Thursday, May 4}
\end{description}

\begin{description}
\item[Attendance Policy]
Attendance is crucial for successful completion of the course, but there
are no grade penalties.
\item[Late Work Policy]
2 day penalty free grace period on all assignments, 5\% flat penalty
afterwards. See
\protect\hyperlink{late-submissions-and-re-submissions}{Late Submissions
and Re-submissions}
\end{description}

\begin{center}\rule{0.5\linewidth}{0.5pt}\end{center}

\hypertarget{course-description}{%
\section{Course Description}\label{course-description}}

In recent decades, there has been a strong ``quantitative turn'' in
linguistics. Quantitative methods, including statistical analysis, have
always been fixtures in some subfields, but there are now few areas of
linguistic inquiry where they are completely absent. As a graduate
course in quantitative methods, the goals of this course are to help you
establish baseline statistical reasoning, and to provide practical
experience in data (re)organization and statistical model building. We
will be focusing our attention on the most common variety of statistical
models (linear models and their generalizations) in the most commonly
used programming language (R).

\begin{center}\rule{0.5\linewidth}{0.5pt}\end{center}

\hypertarget{learning-outcomes}{%
\section{Learning Outcomes}\label{learning-outcomes}}

After attending class meetings and completion of the coursework,
students should be able to

\begin{itemize}
\item
  identify appropriate quantitative analysis procedures for diverse data
  sets.
\item
  organize data sets tidily
\item
  re-organize untidy data sets in R
\item
  generate exploratory data visualizations
\item
  specify and fit linear or generalized linear models in R
\item
  report the meaningful results of a statistical model
\end{itemize}

\begin{center}\rule{0.5\linewidth}{0.5pt}\end{center}

\hypertarget{course-materials}{%
\section{Course Materials}\label{course-materials}}

\hypertarget{required}{%
\subsection{Required:}\label{required}}

Winter, B. (2019). \emph{Statistics for linguists: An introduction using
R}. Routledge. ISBN 978-1138056091.

\hypertarget{recommended}{%
\subsection{Recommended:}\label{recommended}}

Wickham, H \& G. Grolemund (2022*) \emph{R for Data Science}
\url{https://r4ds.had.co.nz/}.

\begin{center}\rule{0.5\linewidth}{0.5pt}\end{center}

\hypertarget{course-technology}{%
\section{Course Technology}\label{course-technology}}

\hypertarget{rrstudio}{%
\subsection{R/RStudio}\label{rrstudio}}

We'll be learning how to implement our analyses in the R programming
language, specifically using the RStudio IDE. You can install and
configure RStudio on your own computer that you bring to class. You will
also have access to RStudio Workbench hosted by the College of Arts \&
Sciences at \url{https://rstudio.as.uky.edu/}. You can log in with your
LinkBlue credentials.

\hypertarget{quartoquarto-notebooks}{%
\subsection{Quarto/Quarto Notebooks}\label{quartoquarto-notebooks}}

Quarto is a program built into RStudio that takes source documents
written in Markdown and R, and renders them into various output document
formats, including html and pdf. This program is included in RStudio,
and won't require additional download or installation.

\hypertarget{gitgithub}{%
\subsection{Git/Github}\label{gitgithub}}

Git is a ``Version Control System'' that lets you keep track of changes
on software projects. Github is a service that allows online hosting of
Git projects. You will need to create a free a Github account for the
course.

\hypertarget{canvas}{%
\subsection{Canvas}\label{canvas}}

Canvas will be used to make course announcements, and to set \& submit
assignments.

\begin{center}\rule{0.5\linewidth}{0.5pt}\end{center}

\hypertarget{communications}{%
\section{Communications}\label{communications}}

I will respond to emails in a timely manner during normal working hours,
but it may take longer if you email me after 5pm on weekdays, or any
time during the weekend.

\begin{center}\rule{0.5\linewidth}{0.5pt}\end{center}

\hypertarget{course-schedule}{%
\section{Course Schedule}\label{course-schedule}}

The topics and readings listed here are the tentative schedule for the
course. We may find, in the room, that some topics will take longer than
initially scheduled.

\footnotesize

\begin{longtable}[]{@{}
  >{\raggedleft\arraybackslash}p{(\columnwidth - 8\tabcolsep) * \real{0.0900}}
  >{\raggedright\arraybackslash}p{(\columnwidth - 8\tabcolsep) * \real{0.1700}}
  >{\raggedright\arraybackslash}p{(\columnwidth - 8\tabcolsep) * \real{0.1900}}
  >{\raggedright\arraybackslash}p{(\columnwidth - 8\tabcolsep) * \real{0.1900}}
  >{\raggedright\arraybackslash}p{(\columnwidth - 8\tabcolsep) * \real{0.3700}}@{}}
\toprule()
\begin{minipage}[b]{\linewidth}\raggedleft
Week
\end{minipage} & \begin{minipage}[b]{\linewidth}\raggedright
Dates
\end{minipage} & \begin{minipage}[b]{\linewidth}\raggedright
Topics
\end{minipage} & \begin{minipage}[b]{\linewidth}\raggedright
Readings
\end{minipage} & \begin{minipage}[b]{\linewidth}\raggedright
Notes
\end{minipage} \\
\midrule()
\endhead
1 & Jan 10,12 & Onboarding &
\href{https://jofrhwld.github.io/blog/posts/2022/12/2022-12-21_github-onboarding/}{Github
Onboarding with RStudio} & Supplementary Resources

\href{https://r4ds.had.co.nz/r-markdown.html}{R4DS: RMarkdown}

\href{https://quarto.org/docs/get-started/hello/rstudio.html}{Quarto
Tutorial: Hello Quarto}

\href{https://quarto.org/docs/get-started/computations/rstudio.html}{Quarto
Tutorial: Computations}

\href{https://quarto.org/docs/get-started/authoring/rstudio.html}{Quarto
Tutorial: Authoring} \\
2 & Jan 17, 19 & Introduction to R & Winter, Ch 1 & Supplementary
Resources:

\href{https://rstudio-education.github.io/hopr/basics.html}{Hands-On
Programming with R, Chapter 2}

\href{https://jofrhwld.github.io/teaching/courses/2017_lsa/lectures/Session_1.nb.html}{2017
LSA Course: Introduction to R} \\
3 & Jan 24, 26 & Tidyverse and Workflows & Winter, Ch 2 & Supplementary
Resources:

\href{https://r4ds.had.co.nz/tidy-data.html}{R4DS, Tidy Data}

\href{https://r4ds.had.co.nz/data-visualisation.html}{R4DS,
Data-visualization}

\href{https://jofrhwld.github.io/teaching/courses/2017_lsa/lectures/Session_2.nb.html}{2017
LSA Course: Data Frames}

\href{https://jofrhwld.github.io/teaching/courses/2017_lsa/lectures/Session_3.nb.html}{2017
LSA Course: Split-Apply-Combine}

\href{https://jofrhwld.github.io/teaching/courses/2017_lsa/lectures/Session_4.nb.html}{2017
LSA Course: ggplot2}

\href{https://posit.co/wp-content/uploads/2022/10/tidyr.pdf}{tidyr cheat
sheet (pdf)}

\href{https://posit.co/wp-content/uploads/2022/10/data-transformation-1.pdf}{dplyr
cheat sheet (pdf)}

\href{https://posit.co/wp-content/uploads/2022/10/data-visualization-1.pdf}{ggplot2
cheat sheet (pdf)} \\
4 & Jan 31, Feb 02 & Tidyverse and Workflows (part 2) & Winter, Ch 2 &
'\,' \\
5 & Feb 07, 09 & Descriptive Statistics, Models, and Distributions &
Winter, Ch3 & Supplementary Resources:

\href{https://r4ds.had.co.nz/exploratory-data-analysis.html}{R4DS,
Exploratory Data Analysis} \\
6 & Feb 14, 16 & Intro to Linear Models & Winter Ch4 & Supplementary
Resources

\href{https://r4ds.had.co.nz/model-basics.html}{R4DS: Model Basics}

\href{https://jofrhwld.github.io/teaching/courses/2017_lsa/lectures/Session_5.nb.html}{2017
LSA Course: Fitting Linear Models} \\
7 & Feb 21, 23 & Correlations and Transformations & Winter Ch5 &
Supplementary Resources

\href{https://r4ds.had.co.nz/model-basics.html}{R4DS: Model Basics}

\href{https://jofrhwld.github.io/teaching/courses/2017_lsa/lectures/Session_5.nb.html}{2017
LSA Course: Fitting Linear Models} \\
8 & Feb 28, Mar 02 & Multiple Regression & Winter Ch6 & Supplementary
Resources

\href{https://r4ds.had.co.nz/model-basics.html}{R4DS: Model Basics}

\href{https://jofrhwld.github.io/teaching/courses/2017_lsa/lectures/Session_5.nb.html}{2017
LSA Course: Fitting Linear Models} \\
9 & Mar 07, 09 & Categorical Predictors & Winter Ch7 & \textbf{Midterm
Project Due}

Supplementary Resources

\href{https://posit.co/wp-content/uploads/2022/10/factors-1.pdf}{forcats
cheat sheet (pdf)} \\
\emph{Spring Break} & \emph{Mar 14, 16} & \emph{No class} & \emph{No
Class} & \\
10 & Mar 21, 23 & Interactions and Non-Linear Effects & Winter Ch 8 &
Supplementary Material

\href{https://arxiv.org/abs/1703.05339}{Sóskuthy (2017)} \\
11 & Mar 28, 30 & Inferential Statistics & Winter Ch 9,10 &
Supplementary Material

Winter Ch 11

\href{https://rpsychologist.com/cohend/}{Visualizing Cohen's \emph{d}}

\href{https://doi.org/10.1016/j.wocn.2018.05.005}{Kirby \& Sonderegger
(2018)} \\
12 & Apr 04, 06 & Generalized Linear Models & Winter Ch 12 &
Supplementary Material

Winter Ch 13 \\
13 & Apr 11, 13 & Mixed Models & Winter Ch 14 & Supplementary Material

\href{https://jofrhwld.github.io/teaching/courses/2017_lsa/lectures/Session_7.nb.html}{2017
LSA Course: Mixed Effects Models} \\
14 & Apr 18, 20 & Mixed Models & Winter Ch 15 & Supplementary Material

\href{https://jofrhwld.github.io/teaching/courses/2017_lsa/lectures/Session_8.nb.html}{2017
LSA Course: Model Comparison and Bootstrapping} \\
15 & Apr 25 & Review \& Outlook & & Winter Ch 16 \\
\emph{Finals} & \emph{May 04} & & & Final Project Due /
\protect\hyperlink{the-final-est-deadline}{The Final-est Deadline} \\
\bottomrule()
\end{longtable}

\normalsize

\begin{center}\rule{0.5\linewidth}{0.5pt}\end{center}

\hypertarget{course-evaluation}{%
\section{Course Evaluation}\label{course-evaluation}}

\begin{figure}

\begin{minipage}[t]{0.50\linewidth}

{\centering 

\hypertarget{grade-components}{%
\subsection{Grade Components}\label{grade-components}}

\begin{longtable}[]{@{}lr@{}}
\toprule()
\endhead
Work-through Blog & 30\% \\
Exercises & 20\% \\
Midterm Project & 20\% \\
Final Project & 20\% \\
Engagement & 10\% \\
\bottomrule()
\end{longtable}

}

\end{minipage}%
%
\begin{minipage}[t]{0.50\linewidth}

{\centering 

\hypertarget{grading-scale}{%
\subsection{Grading Scale}\label{grading-scale}}

\begin{longtable}[]{@{}lr@{}}
\toprule()
\endhead
A & \textgreater= 90 \\
B & 80 to 89 \\
C & 70 to 79 \\
D & 60 to 69 \\
E & \textless= 59 \\
\bottomrule()
\end{longtable}

}

\end{minipage}%

\end{figure}

\hypertarget{assignment-submission}{%
\subsection{Assignment Submission}\label{assignment-submission}}

Assignments will be set on canvas, and you will submit notification of
the assignment's completion through canvas. The \emph{content} of the
assignment itself may be contained elsewhere (e.g.~the A\&S RStudio
server, or a Github repository).

\hypertarget{work-through-blog}{%
\subsection{Work-through Blog}\label{work-through-blog}}

As we work through Winter (2019) chapter by chapter, you will need to
update a Quarto blog in which you, at the very least, run every code
chunk from the chapter. A template Quarto blog repository is already
available on Github. These blog posts will be due at the end of the week
in which we finish covering each chapter.

\hypertarget{exercises}{%
\subsection{Exercises}\label{exercises}}

In addition to running the code included in each chapter, there will
also be occasional R programming exercises.

\hypertarget{midterm-project}{%
\subsection{Midterm Project}\label{midterm-project}}

There will be a midterm project to analyze a sample data set utilizing
the methods covered in the course up to that point, and to report on
your analysis.

\hypertarget{final-project}{%
\subsection{Final Project}\label{final-project}}

There will also be a final project in the same format as the midterm
project, but to extend your analysis tools to the fuller suite of
methods covered in the course.

\hypertarget{engagement}{%
\subsection{Engagement}\label{engagement}}

Inspired by Kirby Conrod's approach to
\href{https://kconrod.medium.com/new-ways-of-encouraging-and-assessing-participation-81928518e518}{Participation
Grades}

\begin{quote}
This portion of the grade is a way for me to give you credit for
informal/unstructured collaborative work that you do. Participation and
collaboration are strong predictors of success and learning retention,
so please make an effort to find a way that works well for you to
participate and engage with your colleagues.
\end{quote}

A well known process for solving programming problems is
``\href{https://en.wikipedia.org/w/index.php?title=Rubber_duck_debugging\&oldid=1111165654}{Rubber
Duck Debugging}.'' It works by describing how each step of a program is
supposed to work to another person or, as the name suggests, a rubber
duck. Often the solution to the problem or the typo causing the bug
jumps out at you during the process. Having a study buddy or study group
could be really helpful if only for this purpose.

\begin{center}\rule{0.5\linewidth}{0.5pt}\end{center}

\hypertarget{late-submissions-and-re-submissions}{%
\section{Late Submissions and
Re-submissions}\label{late-submissions-and-re-submissions}}

Every graded piece of work will have a due date. After a 2 day grace
period, there will be a single, flat 5\% deduction from late work,
whenever it is submitted between the due date and the
\protect\hyperlink{the-final-est-deadline}{The Final-est Deadline}

\hypertarget{midterm-grades}{%
\subsection{Midterm Grades}\label{midterm-grades}}

I will submit midterm grades on March 13, 2023, at the end of the
midterm grading window. Any unsubmitted assignments that were due before
March 13 will be given a grade of 0, \textbf{\emph{BUT}} you can still
submit those assignments after March 13 for their inclusion in the final
grade.

\hypertarget{the-final-est-deadline}{%
\subsection{The Final-est Deadline}\label{the-final-est-deadline}}

The final-est deadline by which to submit any material to be graded is
\textbf{May 4, 2023}. I have to set this hard deadline in order to have
enough time to conclude final grading in time for the university's final
grade submission deadline.

\begin{center}\rule{0.5\linewidth}{0.5pt}\end{center}

\hypertarget{group-work-and-code-sources}{%
\section{Group Work and Code
Sources}\label{group-work-and-code-sources}}

It is acceptable to collaborate and confer with other students in the
course. Any collaboration should be indicated in the assignment
submission. You may also refer to code sources from elsewhere on the
internet, as long as you also document the source, and explain what the
code does. You might not receive credit for code which has been copied
wholesale from another online source or from another student without
credit or documentation.

\hypertarget{large-language-model-a.k.a.-ai-generated-code}{%
\subsection{Large Language Model (a.k.a. AI) Generated
Code}\label{large-language-model-a.k.a.-ai-generated-code}}

There are a number of services that will generate code based on natural
language queries. Some words of warning:

\hypertarget{fluent-bs}{%
\subsubsection{Fluent BS}\label{fluent-bs}}

Large Language Models have been found to generate code that looks
superficially correct, but often does not actually run properly, or do
what the human asker wanted. Being able to successfully identify where
or why code does not work correctly is not always straight forward. This
issue led the Q\&A site StackOverflow to
\href{https://stackoverflow.com/help/gpt-policy}{ban submissions
generated by LLMs}, stating

\begin{quote}
{[}\ldots{]} because GPT is good enough to convince users of the site
that the answer holds merit, signals the community typically use to
determine the legitimacy of their peers' contributions frequently fail
to detect severe issues with GPT-generated answers.
\end{quote}

\hypertarget{explain-what-the-code-does}{%
\subsubsection{Explain what the code
does}\label{explain-what-the-code-does}}

As stated above, you should provide credit to any external sources you
turned to for code help, and explain what the resulting code does.

\begin{center}\rule{0.5\linewidth}{0.5pt}\end{center}

\hypertarget{attendance-and-engagement}{%
\section{Attendance and Engagement}\label{attendance-and-engagement}}

You are expected to attend all scheduled course meetings. It would be
helpful, but not necessary, if you let me know in advance if you are
going to miss any lectures.

If you feel sick in any way, including but not limited to the well-known
symptoms of COVID-19 (loss of taste or smell, a new and persistent
cough, high fever, etc), do not come to class. There are other
mechanisms for demonstrating engagement than attending lectures.

I will also expect all of us in the course to treat each other with
respect and civility in all aspects of the course, including

\begin{itemize}
\item
  In the audio of a Zoom meeting
\item
  In the text chat of a Zoom meeting
\item
  On any course discussion boards or other forums.
\end{itemize}

\begin{center}\rule{0.5\linewidth}{0.5pt}\end{center}

\hypertarget{academic-conduct}{%
\section{Academic Conduct}\label{academic-conduct}}

UK Senate rules on academic offences:
\url{https://www.uky.edu/universitysenate/ao}

Appropriating someone else's work and portraying it as your own is
cheating. Collaborating with someone and portraying that work as solely
your own is cheating. Obtaining answers to homework assignments or exams
from previous semesters is cheating. Using an internet search engine to
look up a question and reporting that answer as your own is cheating.
Falsifying data or experimental results is cheating. If you are unsure
about whether a specific action is cheating, you may check with me.

The minimum penalty for a first offense is a zero on the assignment on
which the offense occurred. If the offense is considered severe or if
the student has other academic offenses on their record, more serious
penalties, up to suspension from the University may be imposed.

When students submit work purporting to be their own, but which in any
way borrows ideas, organization, wording or anything else from another
source without appropriate acknowledgement of the fact, the students are
guilty of plagiarism. Plagiarism includes reproducing someone else's
work, whether it be a published article, chapter of a book, a paper from
a friend or some file, or something similar to this. Plagiarism also
includes the practice of employing or allowing another person to alter
or revise the work which a student submits as their own, whoever that
other person may be.

Students may discuss assignments among themselves or with an instructor
or tutor, but when the actual work is done, it must be done by the
student, and the student alone. When a student's assignment involves
research in outside sources of information, the student must carefully
acknowledge exactly what, where and how they employed them. If the words
of someone else are used, the student must put quotation marks around
the passage in question and add an appropriate indication of its origin.
Making simple changes while leaving the organization, content and
phraseology intact is plagiaristic. However, nothing in these Rules
shall apply to those ideas which are so generally and freely circulated
as to be a part of the public domain (University Senate Rules Section
6.3.1).

\begin{center}\rule{0.5\linewidth}{0.5pt}\end{center}

\hypertarget{university-academic-policy-statements}{%
\section{University Academic Policy
Statements}\label{university-academic-policy-statements}}

Link to University Senate Academic Policy Statements
\url{https://www.uky.edu/universitysenate/acadpolicy}

\hypertarget{excused-absences-and-acceptable-excuses}{%
\subsection{Excused Absences and Acceptable
Excuses}\label{excused-absences-and-acceptable-excuses}}

\textbf{Excused Absences}: \emph{Senate Rules 5.2.5.2.1} defines the
following as acceptable reasons for excused absences: (a) significant
illness, (b) death of a family member, (c) trips for members of student
organizations sponsored by an educational unit, trips for University
classes, and trips for participation in intercollegiate athletic events,
(d) major religious holidays, (e) interviews for graduate/professional
school or full-time employment post-graduation, and (f) other
circumstances found to fit ``reasonable cause for nonattendance'' by the
instructor of record. Students should notify the professor of absences
prior to class when possible.

If a course syllabus requires specific interactions (e.g., with the
instructor or other students), in situations where a student's total
EXCUSED absences exceed 1/5 (or 20\%) of the required interactions for
the course, the student shall have the right to request and receive a
``W,'' or the Instructor of Record may award an ``I'' for the course if
the student declines a ``W.'' (\emph{Senate Rules 5.2.5.2.3.1})

\hypertarget{religious-observances}{%
\subsection{Religious Observances}\label{religious-observances}}

\textbf{Religious Observances}: Students anticipating an absence for a
major religious holiday are responsible for notifying the instructor in
writing of anticipated absences due to their observance of such
holidays. \emph{Senate Rules 5.2.5.2.1(4) requires faculty to include
any notification requirements within the syllabus. If no requirement is
specified, two weeks prior to the absence is reasonable and should not
be given any later. Information regarding major religious holidays may
be obtained through
\href{https://www.uky.edu/ombud/religious-observation-accommodations}{the
Ombud's website}or calling 859-257-3737.}

\hypertarget{verification-of-absences}{%
\subsection{Verification of Absences}\label{verification-of-absences}}

\textbf{Verification of Absences}:Students may be asked to verify their
absences in order for them to be considered excused. \emph{Senate Rule
5.2.5.2.1} states that faculty have the right to request appropriate
verification when students claim an excused absence due to: significant
illness; death in the household, trips for classes, trips sponsored by
an educational unit and trips for participation related to
intercollegiate athletic events; and interviews for full-time job
opportunities after graduation and interviews for graduate and
professional school. (Appropriate notification of absences due to
University-related trips is required prior to the absence when feasible
and in no case more than one week after the absence.)

\hypertarget{make-up-work}{%
\subsection{Make-Up Work}\label{make-up-work}}

\textbf{Make-Up Work}: Students missing any graded work due to an
excused absence are responsible: for informing the Instructor of Record
about their excused absence within one week following the period of the
excused absence (except where prior notification is required); and for
making up the missed work. The instructor must give the student an
opportunity to make up the work and/or the exams missed due to the
excused absence, and shall do so, if feasible, during the semester in
which the absence occurred. The instructor shall provide the student
with an opportunity to make up the graded work and may not simply
calculate the student's grade on the basis of the other course
requirements, unless the student agrees in writing. According to
\emph{SR 5.2.5.2.2}, if a student adds a class after the first day of
classes and misses graded work, the instructor must provide the student
with an opportunity to make up any graded work.

\hypertarget{excused-absences-for-military-duties}{%
\subsection{Excused Absences for Military
Duties}\label{excused-absences-for-military-duties}}

\textbf{Excused Absences for Military Duties}: If a student is required
to be absent for one-fifth or less of the required course interactions
(e.g., class meetings) due to military duties, the following procedure
(per \emph{SR 5.2.5.2.3.2}) shall apply:

\begin{enumerate}
\def\labelenumi{\arabic{enumi}.}
\item
  Once a student is aware of a call to duty, the student shall provide a
  copy of the military orders to the Director of the Veterans Resource
  Center. The student shall also provide the Director with a list of
  his/her courses and instructors.
\item
  The Director will verify the orders with the appropriate military
  authority, and on behalf of the military student, notify each
  Instructor of Record via Department Letterhead as to the known extent
  of the absence.
\item
  The Instructor of Record shall not penalize the student's absence in
  any way and shall provide accommodations and timeframes so that the
  student can make up missed assignments, quizzes, and tests in a
  mutually agreed upon manner.
\end{enumerate}

\hypertarget{unexcused-absences}{%
\subsection{Unexcused Absences}\label{unexcused-absences}}

\textbf{Unexcused Absences:} If an attendance/interaction policy is not
stated in the course syllabus or the policy does not include a penalty
to the student, the instructor cannot penalize a student for any
unexcused absences. (\emph{SR 5.2.5.2.3.3})

\hypertarget{prep-week-and-reading-days}{%
\subsection{Prep Week and Reading
Days}\label{prep-week-and-reading-days}}

\textbf{Prep Week and Reading Days}: Per \emph{Senate Rules 5.2.5.6},
the last week of instruction of a regular semester is termed ``Prep
Week.'' This phrase also refers to the last three days of instruction of
the summer session and winter intersession. The Prep Week rule applies
to ALL courses taught in the fall semester, spring semester, and summer
session, including those taught by distance learning or in a format that
has been compressed into less than one semester or session. This rule
does not apply to courses in professional programs in colleges that have
University Senate approval to have their own calendar.

Make-up exams and quizzes are allowed during Prep Week. In cases of
``Take Home'' final examinations, students shall not be required to
return the completed examination before the regularly scheduled
examination period for that course. No written examinations, including
final examinations, may be scheduled during the Prep Week. No quizzes
may be given during Prep Week. No project/lab
practicals/paper/presentation deadlines or oral/listening examinations
may fall during the Prep Week unless it was scheduled in the syllabus
AND the course has no final examination (or assignment that acts as a
final examination) scheduled during finals week. (A course with a lab
component may schedule the lab practical of the course during Prep Week
if the lab portion does not also require a Final Examination during
finals week.) Class participation and attendance grades are permitted
during Prep Week. The \emph{Senate Rules} permit continuing into Prep
Week regularly assigned graded homework that was announced in the class
syllabus.

For fall and spring semester, the Thursday and Friday of Prep Week are
study days (i.e.~``Reading Days''). There cannot be any required
``interactions'' on a Reading Day. ``Interactions'' include
participation in an in-class or online discussion, attendance at a guest
lecture, or uploading an assignment. See \emph{Senate Rules 9.1} for a
more complete description of required interactions.

\hypertarget{accommodations-due-to-disability}{%
\subsection{Accommodations Due to
Disability}\label{accommodations-due-to-disability}}

\textbf{Accommodations Due to Disability}: In accordance with federal
law, if you have a documented disability that requires academic
accommodations, please inform your instructor as soon as possible during
scheduled office hours. In order to receive accommodations in a course,
you must provide your instructor with a Letter of Accommodation from the
Disability Resource Center (DRC). The DRC coordinates campus disability
services available to students with disabilities. It is located on the
corner of Rose Street and Huguelet Drive in the Multidisciplinary
Science Building, Suite 407. You can reach them via phone at (859)
257-2754,
\href{https://www.uky.edu/universitysenate/drc@uky.edu}{\uline{via
email}} (\href{mailto:drc@uky.edu}{\nolinkurl{drc@uky.edu}}) or visit
their
\href{http://www.uky.edu/DisabilityResourceCenter}{\uline{website}}
(uky.edu/DisabilityResourceCenter). DRC accommodations are not
retroactive and should therefore be established with the DRC as early in
the semester as is feasible.

\hypertarget{non-discrimination-statement-and-title-ix-information}{%
\subsection{Non-Discrimination Statement and Title IX
Information}\label{non-discrimination-statement-and-title-ix-information}}

\textbf{Non-discrimination and Title IX policy}: In accordance with
federal law, UK is committed to providing a safe learning, living, and
working environment for all members of the University community. The
University maintains a comprehensive program which protects all members
from discrimination, harassment, and sexual misconduct. For complete
information about UK's prohibition on discrimination and harassment on
aspects such as race, color, ethnic origin, national origin, creed,
religion, political belief, sex, and sexual orientation, please see
\href{https://www.uky.edu/regs/ar6-1}{\uline{the electronic version of
\emph{UK's Administrative Regulation 6:1} (``Policy on Discrimination
and Harassment'')}} (\url{https://www.uky.edu/regs/ar6-1}). In
accordance with Title IX of the Education Amendments of 1972, the
University prohibits discrimination and harassment on the basis of sex
in academics, employment, and all of its programs and activities. Sexual
misconduct is a form of sexual harassment in which one act is severe
enough to create a hostile environment based on sex and is prohibited
between members of the University community and shall not be tolerated.
For more details, please see
\href{https://www.uky.edu/regs/sites/www.uky.edu.regs/files/files/ar/ar_6.2-interim-final_08_14_20_1.pdf}{\uline{the
electronic version of \emph{Administrative Regulations 6:2} (``Policy
and Procedures for Addressing and Resolving Allegations of Sexual
Harassment Under Title IX and Other Forms of Sexual Misconduct'')}}
(\href{https://www.uky.edu/regs/sites/www.uky.edu.regs/files/files/ar/ar_6.2-interim-final_08_14_20_1.pdf}{https://www.uky.edu/regs/sites/www.uky.edu.regs/files/files/ar/ar\_6.2-in...}).
Complaints regarding violations of University policies on
discrimination, harassment, and sexual misconduct are handled by the
Office of Institutional Equity and Equal Opportunity (Institutional
Equity), which is located in 13 Main Building and can be reached by
phone at (859) 257-8927. You can also visit
\href{https://www.uky.edu/eeo/}{\uline{Institutional Equity's website}}
(\url{https://www.uky.edu/eeo}).

Faculty members are obligated to forward any report made by a student
related to discrimination, harassment, and sexual misconduct to the
Office of Institutional Equity. Students can confidentially report
alleged incidences through the
\href{https://www.uky.edu/vipcenter/}{\uline{Violence Intervention and
Prevention Center}} (\url{https://www.uky.edu/vipcenter}),
\href{https://www.uky.edu/counselingcenter/}{\uline{Counseling Center}}
(\url{https://www.uky.edu/counselingcenter}), or
\href{https://ukhealthcare.uky.edu/university-health-service/student-health}{\uline{University
Health Service}}
(\url{https://ukhealthcare.uky.edu/university-health-service/student-health}).

Reports of discrimination, harassment, or sexual misconduct may be made
via the \href{https://www.uky.edu/eeo/}{\uline{Institutional Equity's
website}} (\url{https://www.uky.edu/eeo}); at that site, click on ``Make
a Report'' on the left-hand side of the page.

\hypertarget{regular-and-substantive-interaction}{%
\subsection{Regular and Substantive
Interaction}\label{regular-and-substantive-interaction}}

\textbf{Regular and Substantive Interaction}: All credit-bearing courses
must support regular and substantive interaction (RSI) between the
students and the instructor, regardless of the course's delivery mode
(e.g., in-person, hybrid, or online). Courses satisfy this requirement
when course participants meet regularly as prescribed in SR 10.6, and
the Instructor of Record substantively interacts with students in at
least two of the following ways: provides direct instruction; assesses
students' learning; provides information or responds to students'
questions; and facilitates student discussions. Some exceptions allowed
as per SACSCOC. For further information about the RSI requirement, see
the \href{https://www.uky.edu/tlai/compliance-faculty}{\uline{Compliance
Resources}} link on the Teaching, Learning and Academic Innovation
Compliance page.

\begin{center}\rule{0.5\linewidth}{0.5pt}\end{center}

\hypertarget{diversity-equity-and-inclusion}{%
\section{Diversity, Equity and
Inclusion}\label{diversity-equity-and-inclusion}}

The University of Kentucky is committed to our core values of diversity
and inclusion, mutual respect and human dignity, and a sense of
community (\href{https://www.uky.edu/regs/gr14}{\uline{Governing
Regulations XIV}}). We acknowledge and respect the seen and unseen
diverse identities and experiences of all members of the university
community (\url{https://www.uky.edu/regs/gr14}). These identities
include but are not limited to those based on race, ethnicity, gender
identity and expressions, ideas and perspectives, religious and cultural
beliefs, sexual orientation, national origin, age, ability, and
socioeconomic status. We are committed to equity and justice and
providing a learning and engaging community in which every member is
engaged, heard, and valued.

We strive to rectify and change behavior that is inconsistent with our
principles and commitment to diversity, equity, and inclusion. If
students encounter such behavior in a course, they are encouraged to
speak with the instructor of record and/or the
\href{https://www.uky.edu/eeo/}{\uline{Office of Institutional Equity
and Equal Opportunity}}. Students may also contact a faculty member
within the department, program director, the director of undergraduate
or graduate studies, the department chair, any college administrator, or
the dean. All of these individuals are mandatory reporters under
University policies.

\url{https://www.uky.edu/universitysenate/syllabus-dei}



\end{document}
